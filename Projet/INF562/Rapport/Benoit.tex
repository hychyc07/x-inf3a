
La première manière et la plus simple de calculer la courbure en un point est la courbure de Gauss. En version discrète, elle s'exprime par la formule suivante :
 \[ \kappa_G(v) = \frac{3(\sum_{i}\theta_{i}-2\pi)}{\sum_{i}  Aire(F_{i})}  \]
Cette formule permet d'obtenir instantanément le produit des courbures principales. Ainsi, en effectuant un passage sur chacun des points on obtient une distribution de la courbure de Gauss du maillage. 

Comme l'on essaye de traiter le problème en étant indépendant par rapport à l'échelle, on impose une certaine moyenne à la distribution des courbures. On divise donc toutes les valeurs par la moyenne désirée, puis grâce à un $ arctan $ on peut transposer l'intervalle non borné des courbures dans un intervalle $ \left[ -1 ; 1 \right]  $.
